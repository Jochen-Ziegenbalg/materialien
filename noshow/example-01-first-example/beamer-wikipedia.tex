\documentclass{beamer}
\usepackage[utf8]{inputenc}
\usetheme{Warsaw}  %% Themenwahl

\title{Präsentation}
\author{Max Mustermann}
\date{\today}

\begin{document}
\maketitle
\frame{\tableofcontents[currentsection]}

\section{Abschnitt 1}
\begin{frame} %%Eine Folie
  \frametitle{Ein Demotitel} %%Folientitel
  \begin{Definition} %%Definition
    Eine Definition
  \end{Definition}
\end{frame}  

\begin{frame} %%Eine Folie  
  \frametitle{Satz des Pythagoras}    %%Folientitel
  \begin{Theorem}    
    Der \textbf{Satz des Pythagoras}  \\ 
    \vspace{2 mm}
    In jedem rechtwinkligen Dreieck mit den Katheten $a$ und $b$ und 
    der Hypotenuse $c$ gilt:   
    $$ a^2 + b^2 = c^2 $$
  \end{Theorem}
\end{frame}

\begin{frame} %%Eine Folie  
  \frametitle{Satz des Pythagoras}    %%Folientitel
    Der \textbf{Satz des Pythagoras}  \\ 
    \vspace{2 mm}
    In jedem rechtwinkligen Dreieck mit den Katheten $a$ und $b$ und 
    der Hypotenuse $c$ gilt:   
    $$ a^2 + b^2 = c^2 $$
\end{frame}

\end{document}